% Formato para un capítulo cualquiera

%Título del capítulo
\chapter{Introducción} 

\section{¿Cómo escribir tu TFG/TFM en \LaTeX{} ?}
Escribir un \ac{TFG} o una \ac{TFM} es una tarea de gran importancia. La herramienta de edición \LaTeX{} te va a permitir editar estos documentos de una forma elegante, útil y segura. Simplemente dedicando unos minutos a comprender cómo están organizados los ficheros que acabas de descargar, y realizando algunas pruebas con ellos, podrás comenzar a escribir el tuyo.

\subsection{La mejor manera para aprender a programar es programando}
\LaTeX{} es un lenguaje de programación como tantos otros a los que estás acostumbrado. Puedes encontrar mucha información y manuales en Internet, siendo el mejor sitio \cite{LatexWiki}. También te recomiendo las referencias siguientes: \cite{Latex1} y \cite{Latex2}.

Este documento PDF que estás leyendo ha sido generado mediante \LaTeX{} utilizando los ficheros:
\begin{enumerate}
\item \textit{plantilla.tex} : es el fichero principal, desde él se hacen las llamadas a los demás ficheros, que pueden ser editados de forma independiente. Si abres este fichero con cualquier editor de textos verás que contiene muchas sentencias que ahora desconoces. Puedes observar que el fichero tiene una zona de cabecera donde se incluyen todos los paquetes a utilizar. Luego se define el título del documento y su autor, para, al final, ir añadiendo los capítulos y secciones de tu \ac{TFG} o \ac{TFM}.
\item \textit{previo.tex}: genera la hoja de calificación oficial para un \ac{TFG} o \ac{TFM} de la Universidad de Alcalá.
\item \textit{dedicatoria.tex}: para que escribas tus dedicatorias.
\item \textit{agradecimientos.tex}: para que escribas los agradecimientos, que seguro son muchos.
\item \textit{resumen.tex}: aquí debes escribir un resumen de tu trabajo.
\item \textit{introduccion.tex}: este es un capítulo modelo, en el que encontrarás los comandos utilizados para generar lo que estás ahora mismo leyendo.
\item \textit{apendice-a.tex}: un modelo de apéndice, muy utilizado en un \ac{TFG} o \ac{TFM} cuando queremos incluir en el documento final código fuente, manuales de usuario, \ldots
\item \textit{bibliografia-pfc.bib} : este es un fichero \textit{.bib} (no lleva la extensión .tex como los anteriores). Es un fichero de bibliografía \textbf{BibTex}. La bibliografía es una parte fundamental de un \ac{TFG}, y es por ello que debemos poner especial cuidado a la hora de editarla, ya que va a permitir que futuros lectores de tu \ac{TFG} o \ac{TFM}, que seguro serán muchos, puedan acudir a las referencias cuando no entiendan algo, o cuando pretendan retomar tu trabajo y continuar con él para mejorarlo. Sobre BibTex también existen muchos manuales, pero encontrarás información útil en \cite{bibtex1}. Para manejar tu bibliografía te recomiendo el programa JabRef\footnote{En Ubuntu está disponible, o si prefieres, puedes descargar la última versión de la página oficial \url{http://jabref.sourceforge.net/}}.
\end{enumerate}

Para probar que todo esto funciona sólo tienes que compilar el fichero \textit{tfc.tex}, ¿pero cómo?. Evidentemente necesitas un compilador. Veamos que opciones existen:

\begin{enumerate}
\item \textbf{\underline{Plataforma Linux (Unix)}}: simplemente necesitas tener instalado el compilador \textit{latex}, que suele estar incluido en un paquete con el mismo nombre. Existen entornos de trabajo bastante agradables y útiles, como son \textit{Kile} o \textit{TexMaker}, sobre los que podrás editar tus documentos de forma cómoda, gráfica y sencilla. También puedes utilizar la herramienta \textit{Lyx}, que te permite saber cómo va quedando tu documento a medida que escribes, sin necesidad de primero editar el código y luego compilar, es decir, es un software de filosofía WYSIWYM (What You See Is What You Mean). O incluso puedes trabajar con editores LaTex online como \textit{Overleaf}.
\item \textbf{\underline{Otras plataformas}}: para trabajar con \LaTeX{} sobre otros sistemas operativos dispones de gran cantidad de software. Simplemente voy a indicarte algunas herramientas que son de libre distribución:
\begin{itemize}
\item Compilador: el único que conozco es \textit{MikTex}, lo puedes descargar de su web oficial.
\item Editor: puedes utilizar TexMaker que es de libre distribución.
\end{itemize}
\end{enumerate}

Una sugerencia: \textbf{\textit{¿no crees que es un buen momento para trabajar desde Linux?}}. Si no tienes este sistema operativo en tu ordenador, prueba a instalar la distribución \textit{Ubuntu} (http://www.ubuntu.com), es realmente sencillo funcionar con ella, y además puedes descargarla desde la web.


Ahora que conoces algunas herramientas, debes probar a compilar el fichero \textit{plantilla.tex} hasta que obtengas como resultado este pdf.

\subsection{Algunos detalles más}

Con \LaTeX{} puedes editar tus propias tablas (Tabla \ref{tabla:primera}), e incluso añadir gráficos a tus documentos (Figura \ref{grafico:primero}). A la hora de añadir un gráfico la mejor opción es trabajar con formatos de imagen \textit{.pdf}, vectorial preferiblemente, aunque puedes incrustar imágenes en formato bmp, jpg y otros muchos.

\begin{table}
\begin{center}
\begin{tabular}{|c|c|c|}\hline
\textbf{Medida} & \textbf{Error} & \textbf{Porcentaje} \% \\ \hline
12 & 23.6 & 22 \\ \hline
-1 & 13 & 4 \\ \hline
6 & 3 & 4 \\ \hline
\end{tabular}
\caption[El título corto de la tabla.]{El título de la tabla.}
\label{tabla:primera}
\end{center}
\end{table}

\begin{figure}
\begin{center}
\includegraphics[width=4cm]{figuras/logo-uah.pdf}\\
\end{center}
\caption[El título corto de la gráfica.]{El título de la gráfica.}
\label{grafico:primero}
\end{figure}


\LaTeX{} también te permite editar ecuaciones de forma muy sencilla y realmente elegante, observa.
\begin{equation}
I = \! \int_{-\infty}^\infty f(x)\,dx \label{eq:fine}.
\end{equation}

\begin{equation}
\label{eq:mdiv}
m(T) =
\begin{cases}
0 & \text{$T > T_c$} \\
\bigl(1 - [\sinh 2 \beta J]^{-4} \bigr)^{\! 1/8} & \text{$T < T_c$}
\end{cases}
\end{equation}

\begin{align}
\textbf{T} &=
\begin{pmatrix}
T_{++} \hfill & T_{+-} \\
T_{-+} & T_{--} \hfill 
\end{pmatrix} , \nonumber \\
& =
\begin{pmatrix}
e^{\beta (J + B)} \hfill & e^{-\beta J} \hfill \\
e^{-\beta J} \hfill & e^{\beta (J - B)} \hfill
\end{pmatrix}.
\end{align}

\section{Recomendaciones importantes finales}
Antes de enviarme como tutor un copia de tu trabajo para revisar, asegúrate de que has realizado las siguientes tareas:
\begin{enumerate}
 \item Pasar un corrector ortográfico. Kile trae uno incorporado. No seguiré revisando ningún documento que contenga más de 3 faltas de ortografía.
 \item Leer lo que hemos escrito y revisarlo hasta que tenga coherencia. No seguiré revisando ningún documento que contenga más de 3 frases que no se entiendan.
 \item Las ecuaciones forman parte del texto, y deben puntuarse. Además, debemos prestar especial atención a las variables que manejamos. No debemos definir dos variables diferentes para representar el mismo concepto. Debe haber consistencia en la formulación matemática.
 \item Todas la figuras que se incluyen deben citarse en el texto. Lo mismo ocurre con las tablas. Debes aprender a manejar los comandos \verb+\label+ y \verb+\ref+.
 \item La Bibliografía es FUNDAMENTAL. Cita bien y cita mucho.  Debes aprender a manejar el comando \verb+\cite+ y a tener una base de datos con todas tus lecturas en formato BibTex.
 \item En la redacción del proyecto procura mantener un estilo serio. Se trata de un documento oficial.
\end{enumerate}


\section{Para terminar}
En el fichero \textit{introduccion.tex} encontrarás todo el código que se ha utilizado para generar este capítulo, échale un vistazo y trata de entender todo aquello que está escrito en él. Si consigues generar este documento pdf de nuevo, es que estás preparado para editar tu propio \ac{TFG} incluyendo los nuevos capítulos que necesites.

\begin{center}
\begin{large}\textbf{Adelante y buena suerte.}\end{large}
\end{center}

